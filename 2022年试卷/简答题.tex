\subsubsection*{三、简答题}
\setcounter{problemname}{0}

\begin{problem}
“元宇宙”是2022年科技界非常火爆的一个概念。你对元宇宙有什么了解吗?请简要描述一下你眼中的元宇宙,以及元宇宙中可能涉及到的三种关键技术,和它们各自承担的作用。
\end{problem}

\begin{solution}
元宇宙是一个使用尖端技术创造无边界创新体验的概念。它包括虚拟现实、芯片、网络通信、人工智能和区块链。虚拟现实让人们体验不同的虚拟世界,芯片优化数字流程,网络通信连接用户,人工智能优化用户体验,区块链确保数字内容的安全和保护。
\end{solution}



\begin{problem}
请简要分析一下图形用户界面取代命令行界面得以广泛流行的主要原因是什么?
\end{problem}

\begin{solution}
GUI 比命令行界面更直观、更具视觉吸引力,并且旨在通过将信息置于特定上下文中,以减少信息单元的数量来最大限度地减少用户记忆。

图形用户界面的优点:
\begin{enumerate}[label=\arabic*.]
    \item 图形用户界面可以直接操纵;
    \item 用户可以在窗口内选取任意交互位置,且不同窗口之间可以叠加;
    \item 鼠标;
    \item 图形界面优于字符界面;(不同的交互方式本身在可用性方面并没有根本性的不同,更重要的是更认真对待界面设计的态度)
\end{enumerate}
\end{solution}



\begin{problem}
在开展某个项目时,某开发人员参考自己的使用习惯对产品的交互设计进行了分析和设计,请分析一下这样做可能存在的问题,以及正确的做法应该如何?
\end{problem}

\begin{solution}
自参考设计:开发人员和实际用户的心智模型可能不同,导致最终产品可能不好用;(局限)

正确做法:充分调研用户,理解你的用户
\end{solution}



\begin{problem}
Mark Weiser 在 ``The computer for the 21st century" 一文中提到要让计算机消失在背景中,请概述你对这句话的理解。\textit{A new way of thinking about computers, one that takes into account the human world and allows the computers themselves to vanish into the background.}
\end{problem}

\begin{solution}
技术应该是不显眼的、直观的和易于使用的。
\end{solution}



\begin{problem}
微信作为今天生活中十分重要的社交工具,其设计中有很多体现“以用户为中心”的设计细节。请列举2个能够体现为用户设计的案例,并进行简要分析。
\end{problem}

\begin{solution}
用户友好的界面、直观的导航和个性化的内容,使访问和享受更容易。

以用户为中心:
\begin{itemize}
    \item 以真实用户和用户目标作为产品开发的驱动力,而不仅仅是以技术为驱动力
    \item 应能充分利用人们的技能和判断力,应支持用户,而不是限制用户
    \item 需要透彻了解用户及用户的任务,并使用这些信息指导设计
    \item 这是一种设计思想,而不是纯粹的技术
\end{itemize}

以用户为中心的设计原则:
\begin{itemize}
    \item 及早以用户为中心:在设计过程的早期就致力于了解用户的需要
    \item 综合设计:设计的所有方面应当齐头并进地发展
    \item 及早并持续性地进行测试:若实际用户认为设计是可行的,它就是可行的
    \item 迭代设计:大问题往往会掩盖小问题的存在
\end{itemize}
\end{solution}



\begin{problem}
请简述Fitts定律,并应用Fitts定律分析比较饼型菜单(Pie Menu)与普通下拉菜单的交互效率。
\end{problem}

\begin{solution}
Fitts定律:预测指点任务的完成时间,时间与目标大小成反比,与和目标的距离成正比。

大目标、小距离具有优势:对选择任务而言,其移动时间随到目标距离的增加而增加,随目标的大小减小而增加
\begin{itemize}
    \item 大目标:屏幕元素应该尽可能多的占据屏幕空间、屏幕元素应尽可能利用屏幕边缘的优势
    \item 小距离:最好的像素是光标所处的像素
\end{itemize}

饼型菜单就是一种典型的大目标,比其他类型的菜单使用更加简单
\end{solution}

