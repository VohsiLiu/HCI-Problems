\subsubsection*{二、填空题}
\setcounter{problemname}{0}

\begin{problem}
以Windows和Mac操作系统为代表的界面范式被称作 \myunderline{图形用户界面}。
\end{problem}


\begin{problem}
帮助我们把相继出现的一组图片组合成一个连续的图像序列,产生动态的影像信息的是\myunderline{感觉记忆(瞬时记忆)}。
\end{problem}


\begin{problem}
在EEC模型中,用户为达目标而制定的动作与系统允许的动作之间的差别被称作\myunderline{执行隔阂}。
\end{problem}



\begin{problem}
在简易可用性工程中,建议将\myunderline{边做边说法}和启发式评估结合使用,可以发现大部分软件可用性问题。
\end{problem}



\begin{problem}
对于主流用户很少使用,但自身需要更新的功能,可使用\myunderline{隐藏}策略进行简化。
\end{problem}

