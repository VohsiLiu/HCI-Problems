\subsubsection*{二、填空题}
\setcounter{problemname}{0}

\begin{problem}
人机交互的英文全称是 \myunderline{Human Computer Interaction}。
\end{problem}


\begin{problem}
记忆的三个阶段分别是\myunderline{感觉记忆}、\myunderline{短时记忆}和\myunderline{长时记忆},其中可比作计算机内存的是\myunderline{短时记忆},存储容量无限的是\myunderline{长时记忆}。
\end{problem}



\begin{problem}
以用户为中心设计方法的英文全称是\myunderline{User Centered Design}。
\end{problem}



\begin{problem}
将页面组件对齐更有助于用户的视觉感知,这主要依据的是格式塔心理学中的\myunderline{连续性}原则。
\end{problem}



\begin{problem}
用户群体中数量最多、最稳定的用户群体是\myunderline{中间用户}。
\end{problem}
