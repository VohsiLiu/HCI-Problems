\subsubsection*{一、选择题}
\setcounter{problemname}{0}

\begin{problem}
	以下哪个领域不会对人机交互学科产生影响:
	\uline{D}    
    \vspace{-0.8em}
    \begin{multicols}{2}
        \begin{enumerate}[label=\Alph*.]
            \item 人机工程学
            \item 认知心理学
            \item 计算机科学
            \item 上述学科均对人机交互学科有影响
        \end{enumerate}
    \end{multicols}
    \vspace{-1em}
\end{problem}



\begin{problem}
	‍以下哪一个不属于可用性目标:
	\uline{D}    
    \vspace{-0.8em}
    \begin{multicols}{4}
        \begin{enumerate}[label=\Alph*.]
            \item 容易学习
            \item 容易使用
            \item 容易发现错误
            \item 使用安全
        \end{enumerate}
    \end{multicols}
    \vspace{-1em}
\end{problem}



\begin{problem}
	‍设计具有类似操作的界面,并为近似任务使用近似变量,使用的是以下哪种原则:
	\uline{B}    
    \vspace{-0.8em}
    \begin{multicols}{4}
        \begin{enumerate}[label=\Alph*.]
            \item 可见性
            \item 一致性
            \item 功能可见性
            \item 以上都不是
        \end{enumerate}
    \end{multicols}
    \vspace{-1em}
\end{problem}



\begin{problem}
	‍可用性目标中应用于表示一个产品在做它应该做的事情方面有多好的指标是:
	\uline{A}    
    \vspace{-0.8em}
    \begin{multicols}{4}
        \begin{enumerate}[label=\Alph*.]
            \item 高效率
            \item 有效性
            \item 可见性
            \item 实用性
        \end{enumerate}
    \end{multicols}
    \vspace{-1em}
\end{problem}



\begin{problem}
	‍以下哪一项不是针对身体有缺陷用户的设计原则:
	\uline{D}    
    \vspace{-0.8em}
    \begin{multicols}{2}
        \begin{enumerate}[label=\Alph*.]
            \item 使用较大文字
            \item 使用文字转语音、语音转文字的转换技术
            \item 鼠标播放到上方时播放语音
            \item 增强现实
        \end{enumerate}
    \end{multicols}
    \vspace{-1em}
\end{problem}



\begin{problem}
	‍以下哪种说法是正确的?
	\uline{C}    
    \vspace{-0.8em}
    \begin{multicols}{2}
        \begin{enumerate}[label=\Alph*.]
            \item 长时记忆的访问速度慢,衰减速度快。
            \item 短时记忆的容量小,衰减慢。
            \item 长时记忆的访问速度慢,衰减速度慢。
            \item 短时记忆的容量大,衰减快。
        \end{enumerate}
    \end{multicols}
    \vspace{-1em}
\end{problem}



\begin{problem}
	‍用户的行为目标和系统允许的完成该目标之间的差异被称作:
	\uline{B}    
    \vspace{-0.8em}
    \begin{multicols}{4}
        \begin{enumerate}[label=\Alph*.]
            \item 评估隔阂
            \item 执行隔阂
            \item 以上两者都对
            \item 以上两者都不对
        \end{enumerate}
    \end{multicols}
    \vspace{-1em}
\end{problem}



\begin{problem}
	‍以下哪一条符合频繁使用的专家用户进行设计的指导原则:
	\uline{D}    
    \vspace{-0.8em}
    \begin{multicols}{2}
        \begin{enumerate}[label=\Alph*.]
            \item 提供指令、对话框和在线帮助
            \item 减轻记忆负担
            \item 提供有意义的信息
            \item 保证快速响应
        \end{enumerate}
    \end{multicols}
    \vspace{-1em}
\end{problem}



\begin{problem}
	‍以下哪一条不是用户测试过程中预先测试一部分:
	\uline{B}    
    \vspace{-0.8em}
    \begin{multicols}{4}
        \begin{enumerate}[label=\Alph*.]
            \item 制定测试计划
            \item 开展小规模测试
            \item 选定测试者
            \item 观察测试者
        \end{enumerate}
    \end{multicols}
    \vspace{-1em}
\end{problem}



\begin{problem}
	以下哪种评估方法不需要真实用户参与:
	\uline{D}    
    \vspace{-0.8em}
    \begin{multicols}{4}
        \begin{enumerate}[label=\Alph*.]
            \item 可用性测试
            \item 协作走查
            \item 问卷调查
            \item 启发式评估
        \end{enumerate}
    \end{multicols}
    \vspace{-1em}
\end{problem}
