\subsubsection*{三、简答题}
\setcounter{problemname}{0}

\begin{problem}
请举例说明什么叫心智模型(Mental Model)?
\end{problem}

\begin{solution}
心智模式又叫心智模型。所谓心智模式是指深植我们心中关于我们自己、别人、组织及周围世界每个层面的假设、形象和故事。并深受习惯思维、定势思维、已有知识的局限。
\end{solution}



\begin{problem}
请简述为什么图形用户界面可以摒弃$7 \pm 2$的设计约束,在界面上放置多个界面组件?
\end{problem}

\begin{solution}
图形用户界面(GUI)相对于命令行界面(CLI)的一个显著优势是其能够放置和管理大量的界面组件,而不受 $7 \pm 2$ 的设计约束。这个数字 $7 \pm 2$ 是由心理学家乔治·米勒提出的米勒定律,指的是人们在短时记忆中能够有效处理的信息单元数量。然而,在GUI中,我们能够摆脱这个限制的原因有几点:

\begin{enumerate}[label=\arabic*.]
    \item \textbf{可视化表示:} GUI通过图形元素的可视化表示,使用户更容易理解和识别界面上的元素。图标、按钮、菜单等图形元素能够通过形状、颜色和位置等特征传达信息,减轻了用户在界面上寻找和理解元素的认知负担。
    \item \textbf{多窗口和分层设计:} GUI允许使用多窗口和分层设计,使用户能够在界面上同时处理多个任务。每个窗口或层次结构中的组件可以独立存在,用户可以方便地切换、最小化或关闭这些窗口。
    \item \textbf{滚动和搜索功能:} GUI提供了滚动和搜索功能,使用户能够查看和访问大量的内容,而不受到屏幕空间的限制。这允许用户在大规模数据集中快速浏览和定位目标。
    \item \textbf{上下文切换:} GUI允许用户通过不同的视图、标签页或面板切换上下文,使得用户能够更有效地组织和管理信息。
\end{enumerate}
\end{solution}



\begin{problem}
简要描述什么是人物角色,以及在其构建时需要注意什么问题?
\end{problem}

\begin{solution}
人物角色是基于观察到的那些真实人的行为和动机,并且在整个设计过程中代表真实的人;是在人口统计学调查收集到的实际用户的行为数据的基础上形成的综合原型。

要注意那些与软件用户界面设计有关的角色特征;要关注使角色之间彼此相区别的特征;要留心焦点角色 (最常见、最典型的角色)。
\end{solution}



\begin{problem}
原型是一种用户乐于接受的需求验证方式,请简要描述一下不同类型的原型在使用时的优缺点。
\end{problem}

\begin{solution}
低保真原型简单、便宜、易于修改,但是和最终产品有一定差距。

高保真原型和最终产品较为接近,但是制作时间长,难以修改,并且容易让用户误以为已经有具体的实现。
\end{solution}



\begin{problem}
请说明Fitts' Law对交互设计有什么启发?
\end{problem}

\begin{solution}
\begin{enumerate}[label=\arabic*.]
    \item 大目标、小距离具有优势
    \vspace{-0.4em}
    \begin{itemize}
        \item 对选择任务而言,其移动时间随到目标距离的增加而增加,随目标的大小减小而增加
    \end{itemize}
    \item 屏幕元素应该尽可能多的占据屏幕空间
    \item 最好的像素是光标所处的像素
    \item 屏幕元素应尽可能利用屏幕边缘的优势
    \item 大菜单,如饼型菜单,比其他类型的菜单使用简单
\end{enumerate}
\end{solution}



\begin{problem}
课上我们为大家介绍了MIT印度裔博士生普拉纳夫设计实现的第六感系统(The SixSense),它可以帮助人们实现更为自然的交互场景。请简要分析一下该系统的硬件组成和核心功能模块。\footnote{\url{https://lcx.cc/post/1550/}}
\end{problem}

\begin{solution}
硬件组成:这套名为“第六感”的设备,由一个网络摄像头、一个微型投影仪附加镜子、一个挂在脖子上的电池包和一台可以上网的3G手机组成。

核心功能模块:将眼前的现实世界变成电脑屏幕,为自己提供数字服务。
\end{solution}



\begin{problem}
在采用观察法进行用户调研时,什么时候可以停止观察?
\end{problem}

\begin{solution}
观察到用户完成任务并确认;用户选择停止任务。
\end{solution}



\begin{problem}
简述一条在他人项目进行启发式评估的作业中发现的一个可用性问题,请简要描述该问题以及其违反的启发式规则。
\end{problem}



\begin{problem}
某设计团队对某个设计问题方案争执不休,最终由公司管理层出面确定了最终方案,请分析他们的做法是否正确,如果不正确请给出你的建议。
\end{problem}

\begin{solution}
不正确,设计问题方案如果出现争执和不确定,应当通过相应的评估手段来解决。
\end{solution}



\begin{problem}
某人计划针对其设计的产品开展评估实验,他根据DECIDE框架设计了实验的各个步骤,然后就开始招募用户进行实验,请简要分析一下他的做法是否正确?
\end{problem}

\begin{solution}
不正确,需要先进行小规模的预实验。
\end{solution}

